\chapter{The Large Hadron Collider and Compact Muon Solenoid experiment}
\label{lhc_and_cms}

\section{The Large Hadron Collider}
The Large Hadron Collider (LHC) collides protons at a center-of-mass energy of $13\TeV$. It was constructed to elucidate the mechanism behind electroweak symmetry breaking, explore physics at the electroweak scale, and search for evidence of BSM physics. The first goal was achieved in 2012 the ATLAS and CMS experiments each announced the discovery of the Higgs boson in 2012 \cite{cms_higgs, atlas_higgs}, and efforts towards the latter two continue today.
% \FIXME{biased away from lhcb and alice?}

The collisions occur at four points around the LHC ring, and each collision point is instrumented with a dedicated detector.

\section{The Compact Muon Solenoid experiment}


CMS uses a right-handed coordinate system centered on the nominal collision point with positive $x$ direction pointing towards the center of the LHC ring and the positive $y$ direction pointing vertically upward. The azimuthal angle in the $x$-$y$ plane, denoted $\phi$, is measured from the positive $x$ axis, and the polar angle $\theta$ is measured from the positive $z$ axis. The angle from the $z$ axis is more commonly described in terms of the pseudorapidity $\eta$, which is defined as $\eta=-\ln\tan(\theta/2)$ \cite{cms_tdr_v1}.

The CMS detector has undergone several upgrades since its initial construction. The description here will focus on the detector conditions relevant to the analysis presented in Section~\ref{displaced_leptons}.

\subsection{Solenoid magnet}
The superconducting solenoid responsible for the 'S' in CMS is designed to produce a \SI{4}{\tesla} magnetic field throughout the \SI{6.3}{\metre} diameter, \SI{12.5}{\metre} long cylindrical volume  that contains the CMS tracker and calorimeters. The magnetic field is produced by running \SI{19}{\kilo\ampere} through 2168 turns of NbTi superconducting cable that are cooled with liquid helium. The flux returns through an iron yoke that also houses the muon system \cite{cms_experiment}. The strong magnetic field is critical to CMS's ability to unambiguously distinguish muons and anti-muons with transverse momenta up to \SI{1}{\TeV} \cite{cms_tdr_v1}.
% probably good to elaborate on physics importance

\subsection{Tracker}
\label{tracker}
In the region closest to the proton collisions, CMS employs a high-granularity silicon tracker to reconstruct particle trajectories and identify primary and secondary vertices. The silicon tracker is subdivided into two regions: inside a radius of \SI{20}{\cm}, the large particle flux demands the use of silicon pixel detectors, while silicon microstrip detectors suffice in the region beyond \SI{20}{\cm}. Furthermore, the pixel detector was replaced between the 2016 and 2017 data-taking periods. As the analysis presented in Section~\ref{displaced_leptons} uses data collected in 2016--2018 and is particularly dependent on tracker measurements, the 2016 (Phase-0) pixel detector, 2017--2018 (Phase-1) pixel detector, and strip detector are described separately below.

\subsubsection{Phase-0 pixel tracker}
\cite{cms_tdr_v1}

\subsubsection{Phase-1 pixel tracker}
\cite{cms_phase1_pixels}

\subsubsection{Strip tracker}
\cite{cms_tdr_v1}

\subsection{Electromagnetic calorimeter}
After traversing the inner tracker, particles next encounter the electromagnetic calorimeter (ECAL). As a homogeneous scintillation calorimeter, ECAL uses \num{61200} lead tungstate crystals in the barrel and \num{7324} in each endcap to reconstruct the energy deposited during electromagnetic showers. Lead tungstate crystals allow for a fast (\SI{80}{\percent} of light emitted within \SI{25}{\nano\s}), compact (radiation length = \SI{0.89}{cm}), fine-grained (Moli\`ere radius = \SI{2.2}{cm}), and radiation hard (up to \SI{10}{\mega rad}) calorimeter. The main drawback is the relatively low light yield (\SI{30}{photon\per\mega\electronvolt}), which necessitates photodetectors with intrinsic gain that work in magnetic fields\cite{cms_experiment, cms_tdr_v1}. Electron energy resolution varies from approximately \num{1}-\SI{5}{\percent} depending on the amount of material traversed before reaching ECAL~\cite{cms_ecal_performance}.

The barrel section extends radially from \num{129} to \SI{177}{cm} and covers up to $|\eta|<1.479$. The crystals are tapered to approximately project back to the IP but not so perfectly that likely particle trajectories align with cracks. Each crystal is approximately one Moli\`ere radius wide and 25 radiation lengths deep. The crystals in each endcap section are arranged in an x-y grid that starts at $\lvert z \rvert = \SI{315}{cm}$ and covers $1.479<\eta<3.0$.

\subsection{Hadronic calorimeter}
Particles that survive the ECAL will next encounter the hadronic calorimeter (HCAL). As the ECAL constitutes approximately 25 radiation lengths but only one interaction length, only particles that decay through the strong force will make it to the HCAL. HCAL is a sampling calorimeter that uses \SI{3.7}{\milli\metre} thick plates of plastic scintillator interspersed within brass absorber to reconstruct the energy deposited during hadronic showers. Embedded wavelength-shifting fibers capture the scintillation light and transfer it to clear fibers to be read out by hybrid photodiodes.

The barrel section ($|\eta|<1.4$) is segmented into \num{32} towers in $\eta$ and \num{64} in $\phi$ that each contain 17 active scintillator layers. In addition, an extra layer (or two at $\eta = \num{0}$) of scintillator tiles sits just outside the solenoid. This extra layer spans covers $|\eta<1.26|$ and increases the minimum effective HCAL interaction length to greater than \num{11.8}.

Each endcap spans a pseudorapidity range of \num{1.3} to \num{3.0} with \num{14} towers in eta and \num{5} to \SI{10}{\degree} $\phi$ segmentation. Also, a steel and quartz fiber forward calorimeter (HF) sits \SI{11.2}{m} from the interaction point and covers $3<|\eta|<5$. In HF, particles produce Cherenkov light when traversing the quartz fibers that run parallel to the beamline.


\subsection{Muon system}
The CMS muon system is composed of three varieties of gaseous detectors embedded in the iron return yoke outside the superconducting solenoid. In the central region ($|\eta|<1.2$), the low muon and neutron rates along with the lower magnetic field, allow the use of drift tube (DT) chambers. At higher $\eta$ ($0.9\leq|\eta|<2.4$), cathode strip chambers (CSCs) are required to handle the higher rates and larger magnetic field. Finally, resistive plate chambers (RPCs), which provide more accurate time measurements and worse spatial resolution than the DTs and CSCs, complement the other detectors out to $|\eta|<1.9$ \cite{cms_tdr_v1, cms_ms_performance}. 

\subsection{Trigger}
\label{trigger}

\subsection{Physics object reconstruction}
%pf, tracking, e, mu