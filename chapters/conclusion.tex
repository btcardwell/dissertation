\chapter{Conclusion}
\label{conclusion}
A search has been presented for new long-lived particles that propagate a measurable distance before decaying to leptons inside the CMS detector. The resulting displaced lepton signature is targeted by selecting events with two leptons (an electron and a muon, two electrons, or two muons) whose transverse impact parameters are between \num{0.01} and \SI{10}{\cm}. The search is performed in 113--118\fbinv of proton-proton collision data produced by the CERN LHC at a center-of-mass energy of 13\TeV and collected by the CMS detector in 2016, 2017, and 2018. This analysis is the first at CMS to target pairs of displaced electrons or muons without requiring that they form a common vertex. The observation is consistent with the background-only hypothesis, and limits are set on the product of the cross section of top squark pair production and the branching fraction to a lepton and a $\cPqb$ or $\cPqd$ quark through an R-parity-violating vertex. For a proper decay length hypothesis of \SI{2}{\cm}, top squarks with masses up \SI{1500}{\GeV} are excluded at the \SI{95}{\percent} confidence level.

Searches for BSM LLPs are critical to exploring the available new-physics parameter space and ultimately to understanding whether new physics exists at currently accessible energy scales. The analysis presented here explicitly constrains the natural parameter space of RPV SUSY models, but more importantly, it also constrains any not-yet-imagined new physics scenarios that could produce displaced leptons. There are still many stones unturned, and the analysis presented in this thesis shines a light on one more region of this unexplored space.

\pagebreak