\chapter{Conclusion}
\label{conclusion}
A search has been presented for new long-lived particles that propagate a measurable distance before decaying to leptons inside the CMS detector. The resulting displaced lepton signature is targeted by selecting events with two leptons (an electron and a muon, two electrons, or two muons) whose transverse impact parameters are between \num{0.01} and \SI{10}{\cm}. The search is performed in 113--118\fbinv of proton-proton collision data produced by the CERN LHC at a center-of-mass energy of 13\TeV and collected by the CMS detector in 2016, 2017, and 2018. This analysis is the first at CMS to target pairs of displaced electrons or muons without requiring that they form a common vertex. The observation is consistent with the background-only hypothesis, and limits are set on the product of the cross section of top squark pair production and the branching fraction to a lepton and a $\cPqb$ or $\cPqd$ quark through an R-parity-violating vertex. For a proper decay length hypothesis of \SI{2}{\cm}, top squarks with masses up \SI{1500}{\GeV} are excluded at the \SI{95}{\percent} confidence level.\fxnote{switch to range?}

Looking to the future, there are several opportunities to increase the sensitivity to new long-lived particles that decay to leptons. Aside from the incremental improvements offered by the expected increases in integrated luminosity and center-of-mass energy provided by the LHC (\SI{14}{\TeV} proton-proton collisions may be available as soon as 2022~\cite{run3_constraints, lhc_schedule}), there are a few changes to the analysis strategy that may be worth pursuing.

The most straightforward improvement would be to study the electron and muon identification requirements with an eye to improving the signal efficiency, especially at large \ad. In particular, the missing inner hit and pixel hit requirements applied to electrons and muons, respectively, in the current analysis effectively limit the maximum LLP decay length to the radius of the CMS pixel detector, which is \SI{16}{\cm}. Any gains in signal efficiency would of course have to be balanced against the likely increase in the mismeasurement background. Thinking along similar lines, it may be interesting to investigate the effects of relaxing the lepton isolation requirement.

A more challenging angle would be to explicitly consider tau leptons in the final state. The analysis presented here is sensitive to displaced taus that decay leptonically to electrons and muons, but a future analysis could likely expand this sensitivity by explicitly studying the \ad behavior of displaced taus. Given the tau decay branching fractions~\cite{pdg_2020}, the largest gain would likely come from considering hadronic tau decays, though this route would also likely represent a considerable challenge.

Finally, one could perform an analysis similar to the one presented here but specifically target new low-mass long-lived particles. The lepton \pt requirements imposed by the trigger limit the low-mass sensitivity of the current analysis. One possible approach would be to adopt a different triggering strategy in the next data-taking period, but it may be that CMS has already collected the ideal dataset in which to perform such a search. In 2018, CMS debuted a novel trigger strategy in which specialized triggers collected approximately ten billion unbiased B-hadron-decay events~\cite{cms_b_parking}. The triggers use a tag-and-probe strategy that actually require the presence of at least one displaced muon whose \pt can be as low as \SI{7}{\GeV}. The trade-off is that most of the muons will be embedded in \cPqb-tagged jets, which will likely necessitate changes to the analysis strategy. Such a search could be an interesting way to cover new ground with existing data.

Searches for BSM LLPs are critical to exploring the available new-physics parameter space and ultimately to understanding whether new physics exists at currently accessible energy scales. The analysis presented here explicitly constrains the natural parameter space of RPV SUSY models, but more importantly, it also constrains any not-yet-imagined new physics scenarios that could produce displaced leptons. There are still many stones unturned, and the analysis presented in this thesis shines a light on one more region of this unexplored space.

\pagebreak