\section{Overview}
\label{overview}



% \FIXME: dumping content that didn't make it into other sections
%This transverse impact parameter (\ad) is defined as the distance of closest approach in the transverse plane of the helical trajectory of the lepton track to the center of the luminous region. Another option of a discriminating variable could be the $d_{0}$ significance $\ad/\sigma$, where $\sigma$ is the uncertainty in \ad. However, we choose to simply use \ad) in this analysis because of its straightforward correspondence to the particle lifetime, which makes it an easier variable for theorists to use in their efforts to reinterpret this search. Furthermore, we found that the uncertainty in \ad is often underestimated, which reduces any potential benefit of using $\ad/\sigma$. Figure \ref{fig:signalEventDiagram} shows a diagram of the definition of $d_0$ used in this analysis. The lepton $d_0$ is strongly correlated with the lifetime of the particle from which it decayed. The $d_0$ spectra of leptons from promptly decaying particles, such as top quarks and $\PW$ bosons, will peak at zero and fall sharply. For leptons from particles that travel farther than the resolution of lepton \ad, such as tau leptons and mesons containing bottom or charm quarks, the $d_0$ spectrum will be noticeably wider. For leptons from much longer-lived particles such as the top squarks in the benchmark signal process, the $d_0$ spectrum will be very wide, approaching a flat distribution for the longest top squark lifetimes we consider. The electron and muon \ad distributions are shown for signals at different lifetimes and backgrounds in Fig.\,\ref{fig:PreselectionD0Signal}, for events that pass the $\Pgm\Pgm$ preselection criteria. For a given lifetime, the \ad distributions are the same for the two $\PSQt$ decay modes considered and are independent of $\PSQt$ mass.

% \FIXME: define dz too

\pagebreak