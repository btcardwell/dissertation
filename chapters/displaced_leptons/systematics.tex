\section{Systematic uncertainties}
\label{systematics}

\subsection{Integrated luminosity}
The integrated luminosities of the 2016, 2017, and 2018 data-taking periods are individually known with uncertainties in the 2.3--2.5\% range~\cite{CMS:lumi2016,CMS:lumi2017,CMS:lumi2018}, while the total Run~2 (2016--2018) integrated luminosity has an uncertainty of 1.8\%, the improvement in precision reflecting the uncorrelated time evolution of some systematic effects. The full table of 2016--2018 integrated luminosity uncertainties are taken from~\cite{lumiTwiki}, with the correlations specified therein.

\subsection{Pileup}
The simulation of pileup events assumes a total inelastic pp cross section of 69.2 mb with an associated uncertainty of 5\%~\cite{Sirunyan:2018nqx}. The systematic uncertainty arising as a result of the modeling of pileup events is estimated by varying the cross section of the minimum bias events by 5\% when generating the target pileup distributions. The pileup weights are recomputed with these new distributions and applied to the simulated events to obtain the variation in the yields in the inclusive signal region. The average uncertainty is between 1 and 2\%. We treat these uncertainties as 100\% correlated across the three years of data taking.

% \FIXME : add tracking efficiency

\subsection{Trigger efficiency}
The trigger efficiency systematic uncertainty is given by the uncertainty in the measured trigger efficiency scale factors (see Section~\ref{trigger_eff}). These uncertainties are 1\% or less for the $\Pe\Pgm$ and $\Pgm\Pgm$ channels and about 10\% for the $\Pe\Pe$ channel.

In addition, we have studied the trigger efficiency in signal as a function of \ad, as shown in Fig.\,\ref{trig_eff_d0}, for events in the trigger \pt plateau. To cover the change observed in the muon trigger efficiency over the full \ad range, we assign an additional 20\% uncertainty.

We treat the trigger efficiency uncertainties as 100\% correlated across the three years of data taking.

\begin{figure}[hbtp]
\centering
\includegraphics[scale=0.3]{figures/systematics/trig_eff/emu_2018/electronAbsD0_100000um_variableBins_coarse.pdf}
\includegraphics[scale=0.3]{figures/systematics/trig_eff/emu_2018/muonAbsD0_100000um_variableBins_coarse.pdf}
\includegraphics[scale=0.3]{figures/systematics/trig_eff/ee_2018/electronAbsD0_100000um_variableBins_coarse.pdf}
\includegraphics[scale=0.3]{figures/systematics/trig_eff/mumu_2018/muonAbsD0_100000um_variableBins_coarse.pdf}
\caption{Trigger efficiency as a function of lepton \ad, for the $\Pe\Pgm$ channel (upper row), the $\Pe\Pe$ channel (lower left), and the $\Pgm\Pgm$ channel (lower right) in 2018 signal, for events in the trigger \pt plateau.}
\label{trig_eff_d0}
\end{figure}

\subsection{Lepton ID and isolation}
To find the systematic uncertainty associated with the corrections to the lepton ID and isolation, we fluctuate the lepton scale factors up and down by their uncertainty and observe the change in the event yields in the inclusive signal region. The average uncertainty for electrons is about 3\% in the $\Pe\Pgm$ channel and about 7\% in the $\Pe\Pe$ channel, while the average uncertainty for muons is $< 1\%$. We treat these uncertainties as 100\% correlated across the three years of data taking.

\subsection{Muon pixel hit efficiency}
The requirement in the muon ID that muons have at least one pixel hit could in principle have some appreciable \ad dependence, so we perform a dedicated study to ensure that we account for any differences in \ad dependence between data and simulation. Figure~\ref{fig:muon_pixel_hit_eff} shows the efficiency of this requirement in cosmic simulation and NoBPTX data as a function of muon \ad. For events in the denominator of these plots, we require  that at least 2 global, PF muons have $\abs{\eta}<1.0$, $\pt>25\GeV$, no displaced vertices in the tracker material, $\az<15\unit{cm}$, and that they pass all the tight ID criteria except the pixel hit requirement. We also require the muons to be separated by $\DR>0.2$. The events in the numerator must pass the same requirements in addition to the requirement that the muons have at least one pixel hit. Using this plot, the mean efficiency to identify the muons in the simulated signal events is evaluated in the same way as done for the displaced tracking efficiency systematic uncertainty. That is, we find the efficiency to identify muons that pass the pixel hit requirement in cosmic simulation and in NoBPTX data, and using the ratio of these two efficiencies, we derive the relative systematic uncertainty in the signal. The average uncertainty is about 16\% (32\%) in the $\Pe\Pgm$ ($\Pgm\Pgm$) channel. As the pixel detector was upgraded after 2016, the 2017 and 2018 systematic uncertainties are treated as fully correlated, while the 2016 uncertainty is treated as uncorrelated with the 2017 and 2018 uncertainties.
% \FIXME : define nobptx etc?, cite displaced tracking uncertainty

\begin{figure}
\centering
\includegraphics[width=0.32\textwidth]{figures/systematics/muon_pixel_hit_eff/muonAbsD0_100000um_variableBins_coarse_2016.pdf}
\includegraphics[width=0.32\textwidth]{figures/systematics/muon_pixel_hit_eff/muonAbsD0_100000um_variableBins_coarse_2017.pdf}
\includegraphics[width=0.32\textwidth]{figures/systematics/muon_pixel_hit_eff/muonAbsD0_100000um_variableBins_coarse_2018.pdf}
\caption{The pixel hit efficiency as a function of muon \ad in simulated cosmic ray events and \texttt{NoBPTX} data in 2016 (left), 2017 (center), and 2018 (right) conditions.}
\label{muon_pixel_hit_eff}
\end{figure}

\subsection{Lepton $d_0$ resolution}
To find the systematic uncertainty associated with the corrections to the lepton $d_0$ (see Section~\ref{d0_smearing}), we fluctuate the lepton $d_0$ corrections up and down by their uncertainty and observe the change in the event yields in the inclusive signal region. The average uncertainty is $<1\%$. We treat these uncertainties as 100\% correlated in 2017 and 2018. No $d_0$ correction or systematic uncertainty is needed for 2016 simulation.

\subsection{Summary of systematic uncertainties in the signal efficiency}
\label{sec:signalSystematicsSummary}

The systematic uncertainties in the signal efficiency are summarized in Table~\ref{tab:systematics}. 

\begin{table}[ht]
\noindent \centering{}\topcaption{\label{tab:systematics}Systematic uncertainties in the signal efficiency for all three years and the three channels. The mean is provided in cases where the uncertainty varies by signal sample. Uncertainties in the same row are treated as correlated among the years of data taking, except for the displaced tracking and muon pixel hit efficiencies, where the 2016 uncertainty is treated as uncorrelated with the 2017 and 2018 uncertainties.}
\begin{tabular}{lrrr}
\hline
Systematic uncertainty & 2016 & 2017 & 2018\tabularnewline
\hline
\textit{Integrated luminosity}      & 1.8\% & 1.8\% & 1.8\%  \tabularnewline
\textit{Pileup} \\ 
 - $\Pe\Pgm$ channel    & 0.5\%   & 0.6\%  & 0.5\%  \tabularnewline
 - $\Pe\Pe$ channel     & 0.5\%   & 0.9\%      & 0.8\%  \tabularnewline
 - $\Pgm\Pgm$ channel   & 0.2\%   & 0.1\%     & 0.2\%  \tabularnewline
\textit{Displaced tracking efficiency}  & 14\%  & 5.8\%   & 2.4\% \tabularnewline
\textit{Trigger efficiency} \\
- $\Pe\Pgm$ channel, electrons   & 1.6\%  & 1.3\%   & 1.2\%   \tabularnewline
- $\Pe\Pgm$ channel, muons       & 1.6\%  & 1.4\%   & 1.2\%   \tabularnewline
- $\Pe\Pe$ channel               & 10\%   & 13\%    & 19\%  \tabularnewline
- $\Pgm\Pgm$ channel             & 1.2\%  & 1.0\%   & 1.1\%   \tabularnewline
\textit{Muon trigger efficiency at large \ad} \\
- $\Pe\Pgm$ channel, muons       & 20\%  & 20\%   & 20\%   \tabularnewline
- $\Pgm\Pgm$ channel             & 20\%  & 20\%   & 20\%   \tabularnewline
\textit{Lepton identification and isolation} \\
- $\Pe\Pgm$ channel, electrons   & 1.2\%     & 3.6\%   & 3.5\%  \tabularnewline
- $\Pe\Pgm$ channel, muons       & 0.05\%  & 0.07\%  & 0.06\% \tabularnewline
- $\Pe\Pe$ channel               & 2.4\%     & 7.2\%    & 7.0\%  \tabularnewline
- $\Pgm\Pgm$ channel             & 0.10\%  & 0.14\%  & 0.12\% \tabularnewline
\textit{Muon pixel hit efficiency} \\
- $\Pe\Pgm$ channel, muons       & 32\%  & 12\%  & 16\% \tabularnewline
- $\Pgm\Pgm$ channel             & 73\%  & 23\%  & 30\% \tabularnewline
\textit{Lepton \ad correction} \\
- $\Pe\Pgm$ channel, electrons     & \NA   & 0.001\%   & 0.001\%  \tabularnewline
- $\Pe\Pgm$ channel, muons         & \NA   & 0.003\%  & 0.001\% \tabularnewline
- $\Pe\Pe$ channel                 & \NA   & 0.11\%   & 0.11\% \tabularnewline
- $\Pgm\Pgm$ channel               & \NA   & 0.11\%   & 0.11\% \tabularnewline
\hline
\end{tabular}
\end{table}

\pagebreak