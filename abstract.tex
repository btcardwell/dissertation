%  The dissertation abstract can only be 500 words.

A search is presented for new long-lived particles that propagate a measurable distance through the CMS detector before decaying to leptons. The search is performed in 113--118\fbinv of proton-proton collision data produced by the CERN LHC at a center-of-mass energy of 13\TeV and collected by the CMS detector in 2016, 2017, and 2018. Events are selected with two leptons (an electron and a muon, two electrons, or two muons) that both have transverse impact parameter values between 0.01\cm and 10\cm. Using transverse impact parameter as the discriminating variable allows for sensitivity to displaced decays without requiring that the leptons form a common vertex. The search is designed to be sensitive to a wide range of new physics models that produce displaced di-lepton final states. The observation is consistent with the background-only hypothesis, and limits are set on the product of the cross-section of top squark pair production and the branching fraction to a lepton and a b or d quark through an R-parity-violating vertex. For a proper decay length hypothesis of \SI{2}{\cm}, top squarks with masses up \SI{1500}{\GeV} are excluded at the \SI{95}{\percent} confidence level.\fxnote{switch to range?}

